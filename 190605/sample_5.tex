\documentclass[a4paper,uplatex]{jsarticle}
\usepackage{hogeo}
\begin{document}
\newcommand{\myouji}{川崎}%ここは\def\myouji{川崎} でもOK!

\newcount\gakunen%\gakunen というカウンタ(数字を入れる箱)を用意
\gakunen=\year%\gakunen に今年(\year)を代入(\yearは,今年の値が入っているカウンタ)。\gakunen=2019となる。
\advance\gakunen-2015%\gakunen から2015を引く。\gakunen=4となる。
\ifnum\the\month<4%今月の数字(\the\month)が4より小さいなら
\relax%何もしない
\else%そうでないなら
\advance\gakunen1%\gakunen に1を足す。
\fi

自己紹介をします。2015年東大入学なので,現在学部の\the\gakunen 年生です。\myouji ほげお(仮名)といいます。

\midashi{\myouji ほげおの趣味}
\begin{itemize}
\item 落語鑑賞
\item カラオケ
\item 水泳
\item 生徒の質問に答えること
\end{itemize}

\midashi{\myouji ほげおのサークル}
\begin{itemize}
\item 東大\TeX 愛好会
\item 東大SATySFi愛好会
\end{itemize}

\midashi{\myouji ほげおの生息場所}
\begin{itemize}
\item 代々木
\item 本郷
\item 駒場
\item 川崎市
\end{itemize}

\midashi{授業内容}

$\noudo{CH_3COOH}$,$\noudo{CH_3COO^-}$,$\noudo{H^+}$に対して$\heikou{CH_3COOH}$は
\[
\heikou{CH_3COOH}=\frac{\noudo{CH_3COO^-}\noudo{H^+}}{\noudo{CH_3COOH}}
\]
と定義される。


\end{document}