\documentclass[b5j,10pt,twoside,uplatex,papersize,landscape]{jsarticle}
\usepackage{fancyhdr}
\usepackage[dvipdfmx]{graphicx,xcolor}
\usepackage{tcolorbox}
\tcbuselibrary{raster}
\tcbuselibrary{skins}
\tcbuselibrary{magazine}

\pagestyle{empty}

%--余白の設定
\setlength{\topmargin}{0mm}
\addtolength{\topmargin}{-1in}
\setlength{\oddsidemargin}{0mm}
\addtolength{\oddsidemargin}{-1in}
\setlength{\evensidemargin}{0mm}
\addtolength{\evensidemargin}{-1in}
\setlength{\textwidth}{\paperwidth}
\addtolength{\textwidth}{0mm}
\setlength{\textheight}{\paperheight}
\addtolength{\textheight}{0mm}
\setlength{\headheight}{0mm}
\setlength{\headsep}{0mm}
\setlength{\footskip}{0mm}
\setlength{\topskip}{0mm}

\begin{document}

%紙の1/2の高さ,1/4の高さを定義しておく
\newdimen\halfpaperheight
\newdimen\quarterpaperheight
\halfpaperheight=\paperheight
\divide\halfpaperheight by2
\quarterpaperheight=\halfpaperheight
\divide\quarterpaperheight by2

%arrayに中身を貯めてゆく。完成形のboxの装飾を意識しながらオプションを書く。
\newboxarray{formentsuke}
\begin{tcolorbox}[%
nobeforeafter,
width=\paperwidth/4,
break at=\halfpaperheight,
height fixed for=all,
enforce breakable,
colframe=white,
colback=white,
sharp corners,
halign=center,
valign=center,
boxrule=0pt,
reset box array=formentsuke,
store to box array=formentsuke,
bottom=10mm,
underlay={\node[above=5mm,font=\footnotesize]
at (frame.south) {- \arabic{tcbbreakpart} -};},
]
\hfil {\Large ここからスタート!}

\newcount\loopcount
\loop あ\advance\loopcount1\ifnum\loopcount<240\repeat\loopcount=0
\loop い\advance\loopcount1\ifnum\loopcount<240\repeat\loopcount=0
\loop う\advance\loopcount1\ifnum\loopcount<240\repeat\loopcount=0
\loop え\advance\loopcount1\ifnum\loopcount<240\repeat\loopcount=0
\loop お\advance\loopcount1\ifnum\loopcount<240\repeat\loopcount=0
\loop か\advance\loopcount1\ifnum\loopcount<240\repeat\loopcount=0
\loop き\advance\loopcount1\ifnum\loopcount<240\repeat\loopcount=0

\hfil {\Large ここまで!}
\end{tcolorbox}

%180度rotate後の1/4高さずれの調整
\mbox{}
\vskip-\quarterpaperheight
\vskip-\baselineskip
\mbox{}
%tcboxedrasterの要領で,外boxの装飾を設定するつもりで書く
\begin{tcbraster}[%
nobeforeafter,
raster columns=4,
raster width=\paperwidth,
raster height=\paperheight,
raster rows=2,
colframe=white,
colback=white,
sharp corners,
halign=center,
valign=center,
boxrule=0pt,
raster before skip=0pt,
raster left skip=0pt,
raster row skip=0pt,
raster right skip=0pt,
raster after skip=0pt,
raster column skip=0pt,
raster equal skip=0pt,
blankest,
]
\begin{tcolorbox}[enhanced,rotate=180]\useboxarray[formentsuke]{1}\end{tcolorbox}
\begin{tcolorbox}[enhanced,rotate=180]\useboxarray[formentsuke]{8}\end{tcolorbox}
\begin{tcolorbox}[enhanced,rotate=180]\useboxarray[formentsuke]{7}\end{tcolorbox}
\begin{tcolorbox}[enhanced,rotate=180]\useboxarray[formentsuke]{6}\end{tcolorbox}
\begin{tcolorbox}\useboxarray[formentsuke]{2}\end{tcolorbox}
\begin{tcolorbox}[enhanced, borderline north={1pt}{-0.5pt}{gray}]\useboxarray[formentsuke]{3}\end{tcolorbox}
\begin{tcolorbox}[enhanced, borderline north={1pt}{-0.5pt}{gray}]\useboxarray[formentsuke]{4}\end{tcolorbox}
\begin{tcolorbox}\useboxarray[formentsuke]{5}\end{tcolorbox}
\end{tcbraster}




\end{document}